\subsection*{Resultados}

Para o Problema do Caixeiro Viajante, executamos cada uma das 12 instâncias do problema 1 vez
e obtivemos os seguintes resultados, onde NNSWP é uma abreviação para a heurística do
Vizinho Mais Próximo com a busca local do Swap e NIOPT para a Inserção Mais Próxima com a busca
local do Or-Opt. A execução foi feita em um computador com uma CPU Intel Core i7 Ultra 155H.

\begin{table}[h!]
    \centering
    \label{tab:exec}
    \begin{tabular}{| c | c | c |}
        \hline
        Instância & NNSWP & NIOPT \\
        \hline
        1 & 2291.4 & 2246.4 \\
        2 & 2355 & 2223 \\
        3 & 1936 & 1753.2 \\
        4 & 1933 & 1810 \\
        5 & 1379.3 & 1352.5 \\
        6 & 1283 & 1448 \\
        7 & 815.5 & 672.7 \\
        8 & 609 & 606 \\
        9 & 477.2 & 438.3 \\
        10 & 403 & 364 \\
        11 & 348.3 & 348.3 \\
        12 & 306 & 305 \\
        \hline
    \end{tabular}
\end{table}

Com base nesta tabela, podemos perceber que a heurística NIOPT costuma produzir soluções
melhores que as NNSWP. Contudo, nem sempre esse resultado será consistente, como nos resultados
da instância 6, onde o NNSWP se sobressaiu. Além disso, em comparação com o algoritmo genético,
que será apresentado no próximo capítulo, nota-se que com instâncias com dados maiores, as
soluções diferem bastante, onde o genético se sobressai em relação às heurísticas deste capítulo.
Por fim, podemos concluir que as heurísticas nem sempre produzirão uma solução ótima, mas
será provavelmente uma solução boa.